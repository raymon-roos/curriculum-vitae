%%%%%%%%%%%%%%%%%%%%%%%%%%%%%%%%%%%%%%%%%
% Freeman Curriculum Vitae
% XeLaTeX Template
% Version 3.0 (September 3, 2021)
%
% This template originates from:
% https://www.LaTeXTemplates.com
%
% Authors:
% Vel (vel@LaTeXTemplates.com)
% Alessandro Plasmati
%
% License:
% CC BY-NC-SA 4.0 (https://creativecommons.org/licenses/by-nc-sa/4.0/)
%
%!TEX program = xelatex
% NOTE: this template must be compiled with XeLaTeX rather than PDFLaTeX
% due to the custom fonts used. The line above should ensure this happens
% automatically, but if it doesn't, your LaTeX editor should have a simple toggle
% to switch to using XeLaTeX.
% 
%%%%%%%%%%%%%%%%%%%%%%%%%%%%%%%%%%%%%%%%%

%----------------------------------------------------------------------------------------
%	PACKAGES AND OTHER DOCUMENT CONFIGURATIONS
%----------------------------------------------------------------------------------------

\documentclass[
	11pt, % Default font size, can be between 8pt and 12pt
]{FreemanCV}

\usepackage[dutch]{babel}

\columnratio{0.50, 0.50} % Widths of the two columns, specified here as a ratio summing to 1 to correspond to percentages; adjust as needed for your content 

% Headers and footers can be added with the following commands: \lhead{}, \rhead{}, \lfoot{} and \rfoot{}
% Example right footer:
%\rfoot{\textcolor{headings}{\sffamily Last update: \today. Typeset with Xe\LaTeX}}

%----------------------------------------------------------------------------------------

\begin{document}

\begin{paracol}{2} % Begin two-column mode

	%----------------------------------------------------------------------------------------
	%	YOUR NAME AND CURRICULUM VITAE TITLE
	%----------------------------------------------------------------------------------------
	\parbox[][0.25\textheight][c]{\linewidth}{ % Box to hold your name and CV title; change the fixed height as needed to match the colored box to the right
		\centering % Horizontally center text
		{\sffamily\Huge Raymon Roos} % Your name
		\medskip % Vertical whitespace

		\vspace{-5pt}
		{\cursivefont\Large\textcolor{headings}{Curriculum Vitae}}
		% \vfill % Push content to the top of the box
	%----------------------------------------------------------------------------------------
	%	YOUR PICTURE
	%----------------------------------------------------------------------------------------

		\begin{center}
			\includegraphics[width=0.21\textwidth,origin=c]{me_myself.jpg}
		\end{center}
		\vspace{-15pt}
	}

	\section{Over mij}

	\textbf{Ik ben...}\\
	...van nature nieuwsgierig, nauwkeurig en analytisch en zoek uitdagingen op. Ik heb een brede 
	interesse, van computer software en hardware; tot online privacy, veiligheid en 
	vrijheid; tot psychologie, filosofie en biologie.

	\textbf{Ik heb geleerd...}\\
	...over wetenschappelijk denken op het Technasium. Op mijn huidige opleiding aan de 
	Bit Academy word ik klaargestoomd voor het actuele werkklimaat in de Tech sector, door 
	middel van praktische en relevante opdrachten en projecten, het werken met Git en 
	vooral het werken in een software team. 

	\textbf{Ik wil uitgedaagd worden...}\\
	...in programmeren voor unieke projecten met een professioneel team, optimaal gebruik maken 
	van Git, de ins en outs leren van frameworks zoals Laravel of Symphony en verdiepen in 
	systeemgericht ontwerpen aan de hand van OOP en FP.

	%----------------------------------------------------------------------------------------
	%	SKILLS DESCRIPTION
	%----------------------------------------------------------------------------------------

	\section{Algemene Vaardigheden}
	\begin{itemize}
		\item nauwkeurig
		\item doelgericht
		\item probleemoplossend vermogen
		\item zelfstandig
        \item taalvaardig \small (Nederlands \& Engels) \normalsize
		\item loyaal
	\end{itemize}

	%----------------------------------------------------------------------------------------
	%	COMPUTER SKILLS
	%----------------------------------------------------------------------------------------
	\section{Computervaardigheden}

	% This section is laid out using a table. A \tableentry command adds
	% lines with the following parameters:
	%\tableentry{Heading}{Content}{spaceafter} All 3 parameters must be
	%supplied but any can be empty if you don't need them A "spaceafter"
	%value in the third parameter will add some vertical space -- this is
	%to be used between headings, leave it empty for no extra space
	%------------------------------------------------
	\begin{supertabular}{r l} % Start a table with two columns, the table will ensure everything is aligned
		%------------------------------------------------
		\tableentry{Beginnend met}{\LaTeX, C, Lua, Laravel, Rust}{}
        \tableentry{}{TailWindCSS}{  }

		%------------------------------------------------
		\tableentry{Enige ervaring}{MySQL, \faHtml5 HTML, JS, OOP}{}
		\tableentry{}{\faLinux GNU/Linux, \faGit GIT, \faCss3 CSS}{  }

		%------------------------------------------------
		\tableentry{Meeste ervaring}{PHP}{}

		%------------------------------------------------
	\end{supertabular}

	\section{Hobbies}

	\begin{itemize}
		\item lezen van klassieke literatuur en Science Fiction
		\item luisteren naar Podcasts en infotainment
		\item programmeren
		\item muziek luisteren
		\item computers en software configureren
	\end{itemize}

	% %----------------------------------------------------------------------------------------
	% %	MAJOR RESEARCH PROJECT
	% %----------------------------------------------------------------------------------------

	% \section{Doctoral Research}

	%  {\raggedright\textbf{``Observation of Einstein-Podolsky-Rosen Entanglement on Supraquantum Structures by Induction Through Nonlinear Transuranic Crystal of Extremely Long Wavelength Pulse from Mode-Locked Source Array"}\par}

	% \medskip % Vertical whitespace

	% My research examined the use of ELW pulses from a mode-locked source array inducted through transuranic crystals to observe entanglement on supraquantum structures. Theoretical advancements included prediction of quantum resonance phenomena including the possibility of resonance cascades. I was motivated to conduct this doctoral research due to my passion for teleportation of matter and I believe I have laid the foundation for further experimental validation and development of practical outcomes.

	% \medskip % Extra vertical whitespace before the next section


	% %----------------------------------------------------------------------------------------
	% %	REFERENCES
	% %----------------------------------------------------------------------------------------

	% \section{References}

	% %\textit{References available on request} % Uncomment if you'd rather not include references and remove the section below

	% %------------------------------------------------

	% % This section is laid out using a table. A \tableentry command adds lines with the following parameters:

	% %\tableentry{Heading}{Content}{spaceafter}
	% % All 3 parameters must be supplied but any can be empty if you don't need them
	% % A "spaceafter" value in the third parameter will add some vertical space -- this is to be used between headings, leave it empty for no extra space

	% %------------------------------------------------

	% \begin{supertabular}{r l} % Start a table with two columns, the table will ensure everything is aligned

	% 	%------------------------------------------------

	% 	\tableentry{}{\textbf{Dr. Isaac Kleiner}}{spaceafter}
	% 	\tableentry{Position}{Professor}{}
	% 	\tableentry{Employer}{\href{https://web.mit.edu/physics/}{Department of Physics}}{}
	% 	\tableentry{}{\href{https://web.mit.edu}{\textit{Massachusetts Institute of Technology}}}{spaceafter}
	% 	\tableentry{Phone}{+1 (617) 253 1000 x5322 (Work)}{}
	% 	\tableentry{Mobile}{+1 (232) 842-3583}{}

	% 	%------------------------------------------------

	% 	\\ % Additional vertical whitespace between the references

	% 	%------------------------------------------------

	% 	\tableentry{}{\textbf{Dr. Eli Vance}}{spaceafter}
	% 	\tableentry{Position}{Scientist (HL1)}{}
	% 	\tableentry{Employer}{\href{http://www.bmrf.us}{Black Mesa Research Facility}}{spaceafter}
	% 	\tableentry{Email}{\href{mailto:e.vance@bmrf.us}{e.vance@bmrf.us}}{}
	% 	\tableentry{Phone}{+1 (800) 786-1410 x6235 (Work)}{}
	% 	\tableentry{Mobile}{+1 (201) 632-3901}{}

	% 	%------------------------------------------------

	% \end{supertabular}
	% \medskip % Extra vertical whitespace before the next section

	% %----------------------------------------------------------------------------------------

	\switchcolumn % Switch to the second (right) column

	%----------------------------------------------------------------------------------------
	%	COLORED CONTACT DETAILS BOX
	%----------------------------------------------------------------------------------------

	\parbox[top][0.14\textheight][c]{\linewidth}{ % Box to hold the colored box; change the fixed height as needed to match the box to the left
		\colorbox{shade}{ % Create colored box and specify background color
			\begin{supertabular}{@{\hspace{3pt}} p{0.05\linewidth} | p{0.775\linewidth}} % Start a table with two columns, the table will ensure everything is aligned
				\color{contactinfo}
				\raisebox{-1pt}{\faHome} & Waver 40A, Ouderkerk a/d Amstel 1191 KJ \\ % Address
				\raisebox{-1pt}{\faPhone} & +31 (0)6 31081991 \\ % Phone number
				\raisebox{-1pt}{\small\faEnvelope} & \href{mailto:raymon.roos@hotmail.com}{raymon.roos@hotmail.com} \\ % Email address
				%\raisebox{-1pt}{\small\faDesktop} & \href{https://www.LaTeXTemplates.com}{https://www.LaTeXTemplates.com} \\ % Website
				\raisebox{-1pt}{\faGithub} & \href{https://github.com/raymon-roos}{github.com/raymon-roos} \\ % GitHub profile
				\raisebox{-1pt}{\faLinkedinSquare} & \href{https://www.linkedin.com/in/raymon-roos-1840a7228/}{linkedin.com/in/raymon-roos} \\ % LinkedIn profile
				% See fontawesome.pdf in the Fonts folder for all icons you can use
			\end{supertabular}
		}
		% \vfill % Push content to the top of the box
	}

	%----------------------------------------------------------------------------------------
	%	EDUCATION
	%----------------------------------------------------------------------------------------

	\section{Opleidingen}

	% Each qualification entry is added with a \qualificationentry command. Below is an empty one to use as a template:

	%\qualificationentry
	%	{} % Duration
	%	{} % Degree
	%	{} % Honors, achievements or distinctions (e.g. first class honors)
	%	{} % Department
	%	{} % Institution

	% All 5 parameters must be supplied but any can be empty if you don't need them

	%------------------------------------------------

	\begin{supertabular}{r l} % Start a table with two columns, the table will ensure everything is aligned

		%------------------------------------------------
		\qualificationentry
		{2021 -- nu} % Duration
		{Software Developer MBO Niv. 4} % Degree
		{} % Honors, achievements or distinctions (e.g. first class honors)
		{Bit Academy} % Department
		{ROC van Amsterdam} % Institution

		%------------------------------------------------
		\qualificationentry
		{2013 -- 2020} % Duration
		{VWO} % Degree
		{Geen diploma} % Honors, achievements or distinctions (e.g. first class honors)
		{Natuur en techniek/Natuur en Gezondheid} % Department
		{Keizer Karel College - Amstelveen} % Institution

		\qualificationentry
		{ } % Duration
		{Technasium Projecten:} % Degree
		{} % Honors, achievements or distinctions (e.g. first class honors)
		{ } % Department
		{ } % Institution
	\end{supertabular}

		\vspace{-28pt}
		\small 
		\begin{itemize}
			\item nieuw dierenverblijf voor dierentuin Artis
			\item kustwering met toegevoegde maatschappelijke waarde voor Ballast Nedam
			\item implementatie van Urban Agriculture voor de gemeente van Amstelveen
			\item methode voor publieke voorlichting over virussen voor Cirion
		\end{itemize}
		\normalsize

	%----------------------------------------------------------------------------------------
	%	WORK EXPERIENCE
	%----------------------------------------------------------------------------------------
	\section{Werkervaring}
	% Each job is added with a \jobentry command. Below is an empty one to use as a template:
	%\jobentry
	%	{} % Duration
	%	{} % FT/PT (full time or part time)
	%	{} % Employer
	%	{} % Job title
	%	{} % Description
	% All 5 parameters must be supplied but any can be empty if you don't need them

	%------------------------------------------------
	\jobentry
	{Feb 2023 -- jun 2023} % Duration
	{FT} % FT/PT (full time or part time)
	{Deboprojects} % Employer
	{Stagiaire backend developer} % Job title
	{Competenties:
		\begin{itemize}
			\item Laravel
		\end{itemize}
	} % Description

	%------------------------------------------------
	\jobentry
	{Sep 2022 -- heden} % Duration
	{PT} % FT/PT (full time or part time)
	{Bit Academy} % Employer
	{Studentencoach} % Job title
	{Competenties:
		\begin{itemize}
			\item vragen beantwoorden
			\item medestudenten helpen
			\item technische problemen oplossen
			\item uitleg geven
		\end{itemize}
	} % Description

	%------------------------------------------------
	\jobentry
	{Aug 2018 -- sep 2022} % Duration
	{PT} % FT/PT (full time or part time)
	{Café-restaurant Bon} % Employer
	{Keukenmedewerker} % Job title
	{Competenties:
		\begin{itemize}
			\item voor- en nagerechten maken
			\item ingrediënten voorbereiden
			\item werkplek onderhouden
			\item bedienen frituur
			\item schoonmaken
			\item hapjes uitdelen
		\end{itemize}
	} % Description


	% %----------------------------------------------------------------------------------------
	% %	AWARDS
	% %----------------------------------------------------------------------------------------

	% \section{Awards}

	% % This section is laid out using a table. A \tableentry command adds lines with the following parameters:

	% %\tableentry{Heading}{Content}{spaceafter}
	% % All 3 parameters must be supplied but any can be empty if you don't need them
	% % A "spaceafter" value in the third parameter will add some vertical space -- this is to be used between headings, leave it empty for no extra space

	% %------------------------------------------------

	% \begin{supertabular}{r l} % Start a table with two columns, the table will ensure everything is aligned

	% 	%------------------------------------------------

	% 	\tableentry{1985}{\textbf{Faculty of Science Masters Scholarship}}{}
	% 	\tableentry{}{\textit{Massachusetts Institute of Technology}}{spaceafter}

	% 	%------------------------------------------------

	% 	\tableentry{1983}{\textbf{Top Achiever Award -- Physics}}{}
	% 	\tableentry{}{\textit{The University of Washington}}{spaceafter}

	% 	%------------------------------------------------

	% \end{supertabular}

	% %----------------------------------------------------------------------------------------
	% %	COMMUNICATION SKILLS
	% %----------------------------------------------------------------------------------------

	% \section{Communication Skills}

	% % This section is laid out using a table. A \tableentry command adds lines with the following parameters:

	% %\tableentry{Heading}{Content}{spaceafter}
	% % All 3 parameters must be supplied but any can be empty if you don't need them
	% % A "spaceafter" value in the third parameter will add some vertical space -- this is to be used between headings, leave it empty for no extra space

	% %------------------------------------------------

	% \begin{supertabular}{r l} % Start a table with two columns, the table will ensure everything is aligned

	% 	%------------------------------------------------

	% 	\tableentry{Conferences}{Oral Presentation at the Annual MIT}{}
	% 	\tableentry{}{Theoretical Physics Conference -- 1987}{spaceafter}

	% 	%------------------------------------------------

	% 	\tableentry{Posters}{Poster at the Meeting of the American}{}
	% 	\tableentry{}{Physical Society -- 1985}{spaceafter}

	% 	%------------------------------------------------

	% \end{supertabular}

	%----------------------------------------------------------------------------------------
	%	Profile DESCRIPTION
	%----------------------------------------------------------------------------------------



	% %----------------------------------------------------------------------------------------
	% %	PUBLICATIONS
	% %----------------------------------------------------------------------------------------

	% \section{Publications}

	% %------------------------------------------------

	% \textbf{Freeman, G. R.} (1996). Chemistry of Multiply Charged Negative Molecular Ions and Clusters in the Gas Phase:  Terrestrial and in Intense Galactic Magnetic Fields. \textit{The Journal of Physical Chemistry}, \textit{100}(11), 4331-4338.

	% \medskip % Vertical whitespace

	% Jacobsen, F. M., Gee, N., \textbf{Freeman, G. R.} (1986). Electron mobility in liquid krypton as function of density, temperature, and electric field strength. \textit{Physical Review A}, \textit{34}(3): 2329-2335.

	% \medskip % Vertical whitespace

	% %------------------------------------------------

	% % As an alternative to a long-form publication list, you can create a shorter summary using only DOI values and years.

	% % Example \doipublication{} command to add another publication:

	% %\doipublication{Year}{DOI}{firstauthor}{spaceafter}

	% % All four parameters are required (can be empty though)
	% % A value of "firstauthor" in the third parameter will output the DOI in bold
	% % A "spaceafter" value in the fourth parameter will add some vertical space -- this is to be used between years

	% %------------------------------------------------

	% \subsection{Publications by DOI}

	% \begin{supertabular}{r l} % Start a table with two columns, the table will ensure everything is aligned

	% 	%------------------------------------------------

	% 	\doipublication{1996}{10.1021/jp951483+}{firstauthor}{spaceafter}

	% 	%------------------------------------------------

	% 	\doipublication{1990}{10.1139/p90-097}{firstauthor}{spaceafter}

	% 	%------------------------------------------------

	% 	\doipublication{1986}{10.1139/v86-297}{}{}
	% 	\doipublication{}{10.1103/PhysRevA.34.2329}{}{spaceafter}

	% 	%------------------------------------------------

	% 	& \textit{First author publications in} \textbf{bold}\\

	% 	%------------------------------------------------

	% \end{supertabular}

	% \medskip % Extra whitespace before the next section

	%----------------------------------------------------------------------------------------

\end{paracol} % End two-column mode

%----------------------------------------------------------------------------------------

\end{document}
